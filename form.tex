\documentclass[15pt]{jarticle}

\usepackage{url}
% 表紙情報
\title{聴覚}
\author{嶺 仁志}
\date{\today}



% 本文
\begin{document}

\maketitle
%-------------------------------------------------------------------------------------------------------
\section{はじめに}
\label{intro}
人間の脳は外部からの様々な刺激を処理している。
私たちは光、音、匂い、味、温度など様々な刺激をうけている。
その中でも、音を受容する「聴覚」について調査する。
%-------------------------------------------------------------------------------------------------------
\section{耳の機能・構造}
\label{ear}
耳の部位を大きく分けると、外側から外耳・中耳・内示の3つに分類することができる。
%---------------------------------------------------------------------------------
\subsection{外耳(耳介~外耳道)}
\label{xxx}
外耳は、耳介と外耳道から構成される。

耳介は、音を集めるための器官で、一般的に「耳」と故障される部位となる。
耳介は前方に向かって突き出していることから、前方からの音を集めやすい構造となっている。
また、音源の方向を特定する音源定位にも役立っている。

外耳道では、次回で集められた尾を共鳴作用により増幅し、鼓膜に伝えられる。
%---------------------------------------------------------------------------------
\subsection{中耳(鼓膜~アブミ骨底)}
\label{yyy}
%---------------------------------------------------------------------------------

\subsection{内耳(蝸牛・三半規管)}
\label{yyy}

\section{音波の処理の流れ}

%-------------------------------------------------------------------------------------------------------
% 読み込み情報
\bibliographystyle{jplain}
\bibliography{reference}

\end{document}