\documentclass[15pt]{jarticle}

\usepackage{url}
% 表紙情報
\title{脳の間隔情報処理}
\author{嶺 仁志}
\date{\today}



% 本文
\begin{document}

\maketitle

\begin{abstract}
脳での感覚情報処理について、特定の一つの間隔を選び、調査せよ

参考にした文献(書籍を含む)等は、レポートに明記(著者、タイトル、発表媒体や出版物名、出版年)してください。

Web 上の情報も参考にしてよいが、雑誌や書籍などの出版情報のある文献を参考文献として必ず含めてください
(Web上の文章のコピー\&ペーストが主となるレポートは作成しないこと)。

分量は、A4で1ページ以上とします。

マイクロソフトWord、またはPDFで、Course Powerを利用し、提出してください。
\end{abstract}


\section{はじめに}
\label{intro}

\section{基礎情報}
\label{basic}
	\subsection{脳}
	\label{brain}
	\subsection{五感}
	\label{five senses}
	五感には視覚、聴覚、触覚、嗅覚、味覚の五つがある。\ref{five senses}
\section{音の概要}

\section{聴覚野}
\cite{聴覚野}

% 読み込み情報
\bibliographystyle{jplain}
\bibliography{reference}

\end{document}