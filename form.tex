\documentclass[12pt]{jarticle}

\usepackage{url}
% 表紙情報
\title{聴覚}
\author{嶺 仁志}
\date{\today}



% 本文
\begin{document}

\maketitle
%-------------------------------------------------------------------------------------------------------
\section{はじめに}
\label{intro}
人間の脳は外部からの様々な刺激を処理している.
私たちは光,音,匂い,味,温度など様々な刺激をうけている.
その中でも,音を受容する「聴覚」について調査する.\cite{構造}
%-------------------------------------------------------------------------------------------------------
\section{耳の機能・構造}
\label{ear}
耳の部位を大きく分けると,外側から外耳・中耳・内示の3つに分類することができる.
%---------------------------------------------------------------------------------
\subsection{外耳(耳介~外耳道)}
\label{outer}
外耳は,耳介と外耳道から構成される.

耳介は,音を集めるための器官で,一般的に「耳」と故障される部位となる.
耳介は前方に向かって突き出していることから,前方からの音を集めやすい構造となっている.
また,音源の方向を特定する音源定位にも役立っている.

外耳道では,次回で集められた尾を共鳴作用により増幅し,鼓膜に伝えられる.
%---------------------------------------------------------------------------------
\subsection{中耳(鼓膜~アブミ骨底)}
\label{midle}
中耳は鼓膜と耳小骨(ツチ骨、キヌタ骨、アブミ骨)から構成される。
外耳道から伝わってきた音は、ツチ骨、キヌタ骨、アブミ骨へと振動を増幅させながら内耳へ伝えていく。
%---------------------------------------------------------------------------------
\subsection{内耳(蝸牛・三半規管)}
\label{inner}
%-------------------------------------------------------------------------------------------------------
\section{聴覚野}
\label{聴覚野}
\ref{ear}章では,感覚器である耳の構造・機能について触れてきた.
\ref{聴覚野}章では耳から受け取った情報を脳でどのように処理されるのかについて触れていく.\cite{聴覚野}

%-------------------------------------------------------------------------------------------------------
% 読み込み情報
\bibliographystyle{jplain}
\bibliography{reference}

\end{document}